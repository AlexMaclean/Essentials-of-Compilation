\documentclass[11pt]{book}
\usepackage[T1]{fontenc}
\usepackage[utf8]{inputenc}
\usepackage{lmodern}
\usepackage{hyperref}
\usepackage{graphicx}
\usepackage[english]{babel}
\usepackage{listings}
\usepackage{amsmath}
\usepackage{amsthm}
\usepackage{amssymb}

%%%%%%%%%%%%%%%%%%%%%%%%%%%%%%%%%%%%%%%%%%%%%%%%%%%%%%%%%%%%%%%%%%%%%%%%%%%%%%%%
% 'dedication' environment: To add a dedication paragraph at the start of book %
% Source: http://www.tug.org/pipermail/texhax/2010-June/015184.html            %
%%%%%%%%%%%%%%%%%%%%%%%%%%%%%%%%%%%%%%%%%%%%%%%%%%%%%%%%%%%%%%%%%%%%%%%%%%%%%%%%
\newenvironment{dedication}
{
   \cleardoublepage
   \thispagestyle{empty}
   \vspace*{\stretch{1}}
   \hfill\begin{minipage}[t]{0.66\textwidth}
   \raggedright
}
{
   \end{minipage}
   \vspace*{\stretch{3}}
   \clearpage
}

%%%%%%%%%%%%%%%%%%%%%%%%%%%%%%%%%%%%%%%%%%%%%%%%
% Chapter quote at the start of chapter        %
% Source: http://tex.stackexchange.com/a/53380 %
%%%%%%%%%%%%%%%%%%%%%%%%%%%%%%%%%%%%%%%%%%%%%%%%
\makeatletter
\renewcommand{\@chapapp}{}% Not necessary...
\newenvironment{chapquote}[2][2em]
  {\setlength{\@tempdima}{#1}%
   \def\chapquote@author{#2}%
   \parshape 1 \@tempdima \dimexpr\textwidth-2\@tempdima\relax%
   \itshape}
  {\par\normalfont\hfill--\ \chapquote@author\hspace*{\@tempdima}\par\bigskip}
\makeatother

%%%%%%%%%%%%%%%%%%%%%%%%%%%%%%%%%%%%%%%%%%%%%%%%%%%

\newcommand{\itm}[1]{\mathit{#1}}
\newcommand{\Atom}{\itm{atom}}
\newcommand{\Stmt}{\itm{stmt}}
\newcommand{\Exp}{\itm{exp}}
\newcommand{\Ins}{\itm{inst}}
\newcommand{\Arg}{\itm{arg}}
\newcommand{\Int}{\itm{int}}
\newcommand{\Var}{\itm{var}}
\newcommand{\Op}{\itm{op}}
\newcommand{\key}[1]{\mathtt{#1}}

%%%%%%%%%%%%%%%%%%%%%%%%%%%%%%%%%%%%%%%%%%%%%%%%%%%
% First page of book which contains 'stuff' like: %
%  - Book title, subtitle                         %
%  - Book author name                             %
%%%%%%%%%%%%%%%%%%%%%%%%%%%%%%%%%%%%%%%%%%%%%%%%%%%

% Book's title and subtitle
\title{\Huge \textbf{Essentials of Compilation} \\ \huge From Scheme to x86 Assembly}
% Author
\author{\textsc{Jeremy G. Siek}
   \thanks{\url{http://homes.soic.indiana.edu/jsiek/}}
   }


\begin{document}

\frontmatter
\maketitle

%%%%%%%%%%%%%%%%%%%%%%%%%%%%%%%%%%%%%%%%%%%%%%%%%%%%%%%%%%%%%%%
% Add a dedication paragraph to dedicate your book to someone %
%%%%%%%%%%%%%%%%%%%%%%%%%%%%%%%%%%%%%%%%%%%%%%%%%%%%%%%%%%%%%%%
\begin{dedication}
Dedicated to PL group at Indiana University.
\end{dedication}

%%%%%%%%%%%%%%%%%%%%%%%%%%%%%%%%%%%%%%%%%%%%%%%%%%%%%%%%%%%%%%%%%%%%%%%%
% Auto-generated table of contents, list of figures and list of tables %
%%%%%%%%%%%%%%%%%%%%%%%%%%%%%%%%%%%%%%%%%%%%%%%%%%%%%%%%%%%%%%%%%%%%%%%%
\tableofcontents
%\listoffigures
%\listoftables

\mainmatter

%%%%%%%%%%%
% Preface %
%%%%%%%%%%%
%\chapter*{Preface}


%\section*{Structure of book}
% You might want to add short description about each chapter in this book.

%\section*{About the companion website}
%The website\footnote{\url{https://github.com/amberj/latex-book-template}} for %this file contains:
%\begin{itemize}
%  \item A link to (freely downlodable) latest version of this document.
%  \item Link to download LaTeX source for this document.
%  \item Miscellaneous material (e.g. suggested readings etc).
%\end{itemize}

%%%%%%%%%%%%%%%%%%%%%%%%%%%%%%%%%%%%
% Give credit where credit is due. %
% Say thanks!                      %
%%%%%%%%%%%%%%%%%%%%%%%%%%%%%%%%%%%%
%\section*{Acknowledgements}
%\mbox{}\\
%\noindent Amber Jain \\
%\noindent \url{http://amberj.devio.us/}

%%%%%%%%%%%%%%%%
% NEW CHAPTER! %
%%%%%%%%%%%%%%%%
\chapter{Integers and Variables}

%\begin{chapquote}{Author's name, \textit{Source of this quote}}
%``This is a quote and I don't know who said this.''
%\end{chapquote}



The $S_0$ language includes integers, operations on integers,
(arithmetic and input), and local variable definitions. This language
is rich enough to exhibit several compilation techniques but simple
enough so that we can reasonably hope to implement the compilation to
x86-64 assembly in two weeks of hard work.  The instructor solution
for the $S_0$ compiler consists of 6 recursive functions and a few
small helper functions that together span 256 lines of code.

The syntax of the $S_0$ language is defined by the following grammar.
\[
\begin{array}{lcl}
  \Op  &::=& \key{+} \mid \key{-} \mid \key{*} \mid \key{read} \\
  \Exp &::=& \Int \mid \Var \mid (\Op \; \Exp^{+}) \mid (\key{let}\, ([\Var \; \Exp])\, \Exp)
\end{array}
\]
As usual, evaluating a Scheme expression produces a value.  For $S_0$,
integers are the only kind of values. To make it straightforward to
map these integers onto x86 assembly, we restrict the integers to just
those representable with 64-bits, the range $-2^{63}$ to $2^{63}$.

The following are a some example expressions in $S_0$ and their value.
\begin{align*}
(+ \; 2 \; 3)  &\Longrightarrow 5 \\
(+ \; 2 \; (- (- 3)))  &\Longrightarrow 5 \\
(\key{let}\,([x \; 3])\, (+ \; 2 \; x)) & \Longrightarrow 5 \\
(\key{let}\,([x \; 3])\, (+ \; (\key{let}\,([x\;2])\, x) \; x)) & \Longrightarrow 5 
\end{align*}
Given user input of 2, the following expression evaluates to $5$.
\[
(+ \; (\key{read}) \; 3)  \Longrightarrow 5
\]
Given user input of 2 then 3, the following expression may evaluate to
either $1$ or $-1$ depending on the whims of the Scheme
implementation.
\[
(+ \; (\key{read}) \; (-\; (\key{read}))) 
\Longrightarrow
1 \text{ or } -1
\]

\section{x86-64 Assembly}

\[
\begin{array}{lcl}
\Arg &::=&  \Int \mid \itm{register} \mid \Int(\itm{register})\\ 
\Ins &::=& \key{addq} \; \Arg \; \Arg \mid 
      \key{subq} \; \Arg \; \Arg \mid 
      \key{imulq} \; \Arg \; \Arg \mid 
      \key{negq} \; \Arg \mid \\
  && \key{movq} \; \Arg \; \Arg \mid 
      \key{callq} \; \mathit{label}
\end{array}
\]

\section{An intermediate C-like language}

\[
\begin{array}{lcl}
\Atom &::=& \Int \mid \Var \\
\Exp &::=& \Atom \mid (\Op \; \Atom^{*})\\
\Stmt &::=& (\key{assign} \; \Var \; \Exp) \mid (\key{return}\; \Exp)
\end{array}
\]


\end{document}
