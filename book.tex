\documentclass[12pt]{book}
\usepackage[T1]{fontenc}
\usepackage[utf8]{inputenc}
\usepackage{lmodern}
\usepackage{hyperref}
\usepackage{graphicx}
\usepackage[english]{babel}
\usepackage{listings}
\usepackage{amsmath}
\usepackage{amsthm}
\usepackage{amssymb}
\usepackage{natbib}
\usepackage{stmaryrd}
\usepackage{xypic}
\usepackage{semantic}

\lstset{%
language=Lisp,
basicstyle=\ttfamily\small,
escapechar=@
}

%%%%%%%%%%%%%%%%%%%%%%%%%%%%%%%%%%%%%%%%%%%%%%%%%%%%%%%%%%%%%%%%%%%%%%%%%%%%%%%%
% 'dedication' environment: To add a dedication paragraph at the start of book %
% Source: http://www.tug.org/pipermail/texhax/2010-June/015184.html            %
%%%%%%%%%%%%%%%%%%%%%%%%%%%%%%%%%%%%%%%%%%%%%%%%%%%%%%%%%%%%%%%%%%%%%%%%%%%%%%%%
\newenvironment{dedication}
{
   \cleardoublepage
   \thispagestyle{empty}
   \vspace*{\stretch{1}}
   \hfill\begin{minipage}[t]{0.66\textwidth}
   \raggedright
}
{
   \end{minipage}
   \vspace*{\stretch{3}}
   \clearpage
}

%%%%%%%%%%%%%%%%%%%%%%%%%%%%%%%%%%%%%%%%%%%%%%%%
% Chapter quote at the start of chapter        %
% Source: http://tex.stackexchange.com/a/53380 %
%%%%%%%%%%%%%%%%%%%%%%%%%%%%%%%%%%%%%%%%%%%%%%%%
\makeatletter
\renewcommand{\@chapapp}{}% Not necessary...
\newenvironment{chapquote}[2][2em]
  {\setlength{\@tempdima}{#1}%
   \def\chapquote@author{#2}%
   \parshape 1 \@tempdima \dimexpr\textwidth-2\@tempdima\relax%
   \itshape}
  {\par\normalfont\hfill--\ \chapquote@author\hspace*{\@tempdima}\par\bigskip}
\makeatother

%%%%%%%%%%%%%%%%%%%%%%%%%%%%%%%%%%%%%%%%%%%%%%%%%%%%%%%%%%%%%%%%%%%%%%%%%%%%%%%%

\newcommand{\itm}[1]{\ensuremath{\mathit{#1}}}
\newcommand{\Stmt}{\itm{stmt}}
\newcommand{\Exp}{\itm{exp}}
\newcommand{\Instr}{\itm{instr}}
\newcommand{\Prog}{\itm{prog}}
\newcommand{\Arg}{\itm{arg}}
\newcommand{\Int}{\itm{int}}
\newcommand{\Var}{\itm{var}}
\newcommand{\Op}{\itm{op}}
\newcommand{\key}[1]{\texttt{#1}}
\newcommand{\READ}{(\key{read})}
\newcommand{\UNIOP}[2]{(\key{#1}\,#2)}
\newcommand{\BINOP}[3]{(\key{#1}\,#2\,#3)}
\newcommand{\LET}[3]{(\key{let}\,([#1\;#2])\,#3)}

\newcommand{\ASSIGN}[2]{(\key{assign}\,#1\;#2)}
\newcommand{\RETURN}[1]{(\key{return}\,#1)}

\newcommand{\INT}[1]{(\key{int}\;#1)}
\newcommand{\REG}[1]{(\key{reg}\;#1)}
\newcommand{\VAR}[1]{(\key{var}\;#1)}
\newcommand{\STACKLOC}[1]{(\key{stack}\;#1)}

\newcommand{\IF}[3]{(\key{if}\,#1\;#2\;#3)}

%%%%%%%%%%%%%%%%%%%%%%%%%%%%%%%%%%%%%%%%%%%%%%%%%%%%%%%%%%%%%%%%%%%%%%%%%%%%%%%%

\title{\Huge \textbf{Essentials of Compilation} \\ 
  \huge An Incremental Approach}

\author{\textsc{Jeremy G. Siek} \\
%\thanks{\url{http://homes.soic.indiana.edu/jsiek/}} \\
  Indiana University \\
  \\
  with contributions from: \\
  Carl Factora
   }

\begin{document}

\frontmatter
\maketitle

\begin{dedication}
This book is dedicated to the programming language wonks at Indiana
University.
\end{dedication}

\tableofcontents
%\listoffigures
%\listoftables

\mainmatter

%%%%%%%%%%%%%%%%%%%%%%%%%%%%%%%%%%%%%%%%%%%%%%%%%%%%%%%%%%%%%%%%%%%%%%%%%%%%%%%%
\chapter*{Preface}

Talk about nano-pass \citep{Sarkar:2004fk,Keep:2012aa} and incremental
compilers \citep{Ghuloum:2006bh}.

%\section*{Structure of book}
% You might want to add short description about each chapter in this book.

%\section*{About the companion website}
%The website\footnote{\url{https://github.com/amberj/latex-book-template}} for %this file contains:
%\begin{itemize}
%  \item A link to (freely downlodable) latest version of this document.
%  \item Link to download LaTeX source for this document.
%  \item Miscellaneous material (e.g. suggested readings etc).
%\end{itemize}

\section*{Acknowledgments}

Need to give thanks to 
\begin{itemize}
\item Kent Dybvig
\item Daniel P. Friedman
\item Abdulaziz Ghuloum
\item Oscar Waddell
\item Dipanwita Sarkar
\item Ronald Garcia
\item Bor-Yuh Evan Chang
\end{itemize}

%\mbox{}\\
%\noindent Amber Jain \\
%\noindent \url{http://amberj.devio.us/}

%%%%%%%%%%%%%%%%%%%%%%%%%%%%%%%%%%%%%%%%%%%%%%%%%%%%%%%%%%%%%%%%%%%%%%%%%%%%%%%%
\chapter{Abstract Syntax Trees, Matching, and Recursion}
\label{ch:trees-recur}

\section{Abstract Syntax Trees}

\marginpar{\scriptsize Introduce s-expressions, quote, and quasi-quote, and comma in
  this section. Make sure to include examples of ASTs. The description
  here of grammars is incomplete. It doesn't really say what grammars are or what they do, it
  just shows an example. I would recommend reading my blog post: a crash course on
  notation in PL theory, especially the sections on Definition by Rules
  and Language Syntax and Grammars. -JGS}
\marginpar{\scriptsize The lambda calculus is more complex of an example that what we really
  need at this point. I think we can make due with just integers and arithmetic. -JGS}
% \begin{enumerate}
% \item language representation
% \item reading grammars
% \end{enumerate}
Abstract syntax trees (AST) are used to represent and model the syntax of a
language. In compiler implementation, we use them to represent intermediary 
languages (IL). Representing ILs with ASTs allow us to categorize expressions
our language along with the restricting the context in which they can 
appear. A simple example is the representation of the untyped 
\mbox{\(\lambda\)-calculus} with simple arithmetic operators. For our 
purposes, we use Racket syntax.

\begin{verbatim}
op  ::= + | - | *
exp ::= n | (op exp*) | x | (lambda (x) exp) | (exp exp)
\end{verbatim}
\marginpar{\scriptsize Regarding de-Bruijnizing as an example... that strikes me
  as something that may be foreign to many readers. The examples in this
  first chapter should try to be simple and hopefully connect with things
  that the reader is already familiar with. -JGS}
With this specification, we can more easily perform \textit{syntax 
transformations} on any expression within the given language (i.e., 
\(\lambda\)-calculus). In the above AST, the syntax {\tt exp*} signifies
\textit{zero or more} {\tt exp}. Later on in this chapter,  we show how 
to transform an arbitrary \(\lambda\)-term into the equivalent 
\textit{de-Bruijinized} \(\lambda\)-term.


\section{Using Match}
% \begin{enumerate}
% \item Syntax transformation
% \item Some Racket examples (factorial?)
% \end{enumerate}

Racket provides a built-in pattern-matcher, {\tt match}, that we can use to
perform syntax transformations. As a preliminary example, we include a
familiar definition of factorial, without using match.
\begin{verbatim}
(define (! n)
  (if (zero? n) 1
      (* n (! (sub1 n)))))
\end{verbatim}
In this form of factorial, we are simply conditioning on the inputted
natural number, {\tt n}. If we rewrite factorial to use {\tt match}, we can
match on the actual value of {\tt n}.
\begin{verbatim}
(define (! n)
  (match n
    (0 1)
    (n (* n (! (sub1 n))))))
\end{verbatim}
Of course, we can also use {\tt match} to pattern match on more complex
expressions.

If we were told to write a function that takes a \(\lambda\)-term as input,
we can match on the values of \textit{syntax-expressions} ({\tt sexp}). We
can then represent the language of Figure ?? with the following function
that uses {\tt match}.
\begin{verbatim}
(lambda (exp)
  (match exp
    ((? number?) ...)
    ((? symbol?) ...)
    (`(,op exp* ...)
     #:when (memv op '(+ - *))
     ...)
    (`(lambda (,x) ,b) ...)
    (`(,e1 ,e2) ...)))
\end{verbatim}
It's easy to get lost in Racket's {\tt match} syntax. To understand this,
we can represent the possible ways of writing \textit{left-hand side} (LHS)
match expressions.
\begin{verbatim}
exp ::= val | (unquote val) | (exp exp*)
lhs ::= val | (quote val*) | (quasi-quote exp) | (? Racket-pred) 
\end{verbatim}

\section{Recursion}
% \begin{enumerate}
% \item \textit{What is a base case?}
% \item Using on a language (lambda calculus -> 
% \end{enumerate}

%%%%%%%%%%%%%%%%%%%%%%%%%%%%%%%%%%%%%%%%%%%%%%%%%%%%%%%%%%%%%%%%%%%%%%%%%%%%%%%%
\chapter{Integers and Variables}
\label{ch:int-exp}

%\begin{chapquote}{Author's name, \textit{Source of this quote}}
%``This is a quote and I don't know who said this.''
%\end{chapquote}

\section{The $S_0$ Language}

The $S_0$ language includes integers, operations on integers,
(arithmetic and input), and variable definitions.  The syntax of the
$S_0$ language is defined by the grammar in
Figure~\ref{fig:s0-syntax}. This language is rich enough to exhibit
several compilation techniques but simple enough so that we can
implement a compiler for it in two weeks of hard work.  To give the
reader a feeling for the scale of this first compiler, the instructor
solution for the $S_0$ compiler consists of 6 recursive functions and
a few small helper functions that together span 256 lines of code.

\begin{figure}[htbp]
\centering
\fbox{
\begin{minipage}{0.85\textwidth}
\[
\begin{array}{lcl}
  \Op  &::=& \key{+} \mid \key{-} \mid \key{*} \mid \key{read} \\
  \Exp &::=& \Int \mid (\Op \; \Exp^{*}) \mid \Var \mid \LET{\Var}{\Exp}{\Exp}
\end{array}
\]
\end{minipage}
}
\caption{The syntax of the $S_0$ language. The abbreviation \Op{} is
  short for operator, \Exp{} is short for expression, \Int{} for integer,
  and \Var{} for variable.}
\label{fig:s0-syntax}
\end{figure}

The result of evaluating an expression is a value.  For $S_0$, values
are integers. To make it straightforward to map these integers onto
x86-64 assembly~\citep{Matz:2013aa}, we restrict the integers to just
those representable with 64-bits, the range $-2^{63}$ to $2^{63}$.

We will walk through some examples of $S_0$ programs, commenting on
aspects of the language that will be relevant to compiling it.  We
start with one of the simplest $S_0$ programs; it adds two integers.
\[
\BINOP{+}{10}{32}
\]
The result is $42$, as you might expected. 
%
The next example demonstrates that expressions may be nested within
each other, in this case nesting several additions and negations.
\[
\BINOP{+}{10}{ \UNIOP{-}{ \BINOP{+}{12}{20} } }
\]
What is the result of the above program?

The \key{let} construct stores a value in a variable which can then be
used within the body of the \key{let}. So the following program stores
$32$ in $x$ and then computes $\BINOP{+}{10}{x}$, producing $42$.
\[
\LET{x}{ \BINOP{+}{12}{20} }{ \BINOP{+}{10}{x} } 
\]
When there are multiple \key{let}'s for the same variable, the closest
enclosing \key{let} is used. Consider the following program with two
\key{let}'s that define variables named $x$.
\[
\LET{x}{32}{ \BINOP{+}{ \LET{x}{10}{x} }{ x } }
\]
For the purposes of showing which variable uses correspond to which
definitions, the following shows the $x$'s annotated with subscripts
to distinguish them.
\[
\LET{x_1}{32}{ \BINOP{+}{ \LET{x_2}{10}{x_2} }{ x_1 } }
\]

The \key{read} operation prompts the user of the program for an
integer. Given an input of $10$, the following program produces $42$.
\[
\BINOP{+}{(\key{read})}{32}
\]
We include the \key{read} operation in $S_0$ to demonstrate that order
of evaluation can make a different. Given the input $52$ then $10$,
the following produces $42$ (and not $-42$).
\[
\LET{x}{\READ}{ \LET{y}{\READ}{ \BINOP{-}{x}{y} } }
\]
The initializing expression is always evaluated before the body of the
\key{let}, so in the above, the \key{read} for $x$ is performed before
the \key{read} for $y$.
%
The behavior of the following program is somewhat subtle because
Scheme does not specify an evaluation order for arguments of an
operator such as $-$.
\[
\BINOP{-}{\READ}{\READ}
\]
Given the input $42$ then $10$, the above program can result in either
$42$ or $-42$, depending on the whims of the Scheme implementation.

The goal for this chapter is to implement a compiler that translates
any program $p \in S_0$ into a x86-64 assembly program $p'$ such that
the assembly program exhibits the same behavior on an x86 computer as
the $S_0$ program running in a Scheme implementation.
\[
\xymatrix{
p \in S_0  \ar[rr]^{\text{compile}} \ar[drr]_{\text{run in Scheme}\quad}   &&  p' \in \text{x86-64} \ar[d]^{\quad\text{run on an x86 machine}}\\
& & n \in \mathbb{Z}   
}
\]
In the next section we introduce enough of the x86-64 assembly
language to compile $S_0$.

\section{The x86-64 Assembly Language}

An x86-64 program is a sequence of instructions. The instructions
manipulate 16 variables called \emph{registers} and can also load and
store values into \emph{memory}. Memory is a mapping of 64-bit
addresses to 64-bit values. The syntax $n(r)$ is used to read the
address $a$ stored in register $r$ and then offset it by $n$ bytes (8
bits), producing the address $a + n$. The arithmetic instructions,
such as $\key{addq}\,s\,d$, read from the source $s$ and destination
argument $d$, apply the arithmetic operation, then stores the result
in the destination $d$. In this case, computing $d \gets d + s$.  The
move instruction, $\key{movq}\,s\,d$ reads from $s$ and stores the
result in $d$. The $\key{callq}\,\mathit{label}$ instruction executes
the procedure specified by the label, which we shall use to implement
\key{read}. Figure~\ref{fig:x86-a} defines the syntax for this subset
of the x86-64 assembly language.

\begin{figure}[tbp]
\fbox{
\begin{minipage}{0.96\textwidth}
\[
\begin{array}{lcl}
\itm{register} &::=& \key{rsp} \mid \key{rbp} \mid \key{rax} \mid \key{rbx} \mid \key{rcx}
              \mid \key{rdx} \mid \key{rsi} \mid \key{rdi} \mid \\
              && \key{r8} \mid \key{r9} \mid \key{r10}
              \mid \key{r11} \mid \key{r12} \mid \key{r13}
              \mid \key{r14} \mid \key{r15} \\
\Arg &::=&  \key{\$}\Int \mid \key{\%}\itm{register} \mid \Int(\key{\%}\itm{register}) \\ 
\Instr &::=& \key{addq} \; \Arg, \Arg \mid 
      \key{subq} \; \Arg, \Arg \mid 
      \key{imulq} \; \Arg,\Arg \mid 
      \key{negq} \; \Arg \mid \\
  && \key{movq} \; \Arg, \Arg \mid 
      \key{callq} \; \mathit{label} \mid
      \key{pushq}\;\Arg \mid \key{popq}\;\Arg \mid \key{retq} \\
\Prog &::= & \key{.globl \_main}\\
      &    & \key{\_main:} \; \Instr^{+}
\end{array}
\]
\end{minipage}
}
\caption{A subset of the x86-64 assembly language.}
\label{fig:x86-a}
\end{figure}

Figure~\ref{fig:p0-x86} depicts an x86-64 program that is equivalent
to $\BINOP{+}{10}{32}$. The \key{globl} directive says that the
\key{\_main} procedure is externally visible, which is necessary so
that the operating system can call it. The label \key{\_main:}
indicates the beginning of the \key{\_main} procedure.  The
instruction $\key{movq}\,\$10, \%\key{rax}$ puts $10$ into the
register \key{rax}. The following instruction $\key{addq}\,\key{\$}32,
\key{\%rax}$ adds $32$ to the $10$ in \key{rax} and puts the result,
$42$, back into \key{rax}. The instruction \key{retq} finishes the
\key{\_main} function by returning the integer in the \key{rax}
register to the operating system.

\begin{figure}[htbp]
\centering
\begin{minipage}{0.6\textwidth}
\begin{lstlisting}
	.globl _main
_main:
	movq	$10, %rax
	addq	$32, %rax
	retq
\end{lstlisting}
\end{minipage}
\caption{A simple x86-64 program equivalent to $\BINOP{+}{10}{32}$.}
\label{fig:p0-x86}
\end{figure}

The next example exhibits the use of memory.  Figure~\ref{fig:p1-x86}
lists an x86-64 program that is equivalent to $\BINOP{+}{52}{
  \UNIOP{-}{10} }$. To understand how this x86-64 program uses memory,
we need to explain a region of memory called called the
\emph{procedure call stack} (\emph{stack} for short). The stack
consists of a separate \emph{frame} for each procedure call. The
memory layout for an individual frame is shown in
Figure~\ref{fig:frame}.  The register \key{rsp} is called the
\emph{stack pointer} and points to the item at the top of the
stack. The stack grows downward in memory, so we increase the size of
the stack by subtracting from the stack pointer. The frame size is
required to be a multiple of 16 bytes. The register \key{rbp} is the
\emph{base pointer} which serves two purposes: 1) it saves the
location of the stack pointer for the procedure that called the
current one and 2) it is used to access variables associated with the
current procedure. We number the variables from $1$ to $n$. Variable
$1$ is stored at address $-8\key{(\%rbp)}$, variable $2$ at
$-16\key{(\%rbp)}$, etc.

\begin{figure}
\centering
\begin{minipage}{0.6\textwidth}
\begin{lstlisting}
	.globl _main
_main:
	pushq	%rbp
	movq	%rsp, %rbp
	subq	$16, %rsp

	movq	$10, -8(%rbp)
	negq	-8(%rbp)
	movq	$52, %rax
	addq	-8(%rbp), %rax

	addq	$16, %rsp
	popq	%rbp
	retq
\end{lstlisting}
\end{minipage}
\caption{An x86-64 program equivalent to $\BINOP{+}{52}{\UNIOP{-}{10} }$.}
\label{fig:p1-x86}
\end{figure}


\begin{figure}
\centering
\begin{tabular}{|r|l|} \hline
Position & Contents \\ \hline
8(\key{\%rbp}) & return address \\
0(\key{\%rbp}) & old \key{rbp} \\
-8(\key{\%rbp}) & variable $1$ \\
-16(\key{\%rbp}) & variable $2$ \\
 \ldots  & \ldots \\
0(\key{\%rsp}) & variable $n$\\ \hline
\end{tabular}

\caption{Memory layout of a frame.}
\label{fig:frame}
\end{figure}

Getting back to the program in Figure~\ref{fig:p1-x86}, the first
three instructions are the typical prelude for a procedure.  The
instruction \key{pushq \%rbp} saves the base pointer for the procedure
that called the current one onto the stack and subtracts $8$ from the
stack pointer. The second instruction \key{movq \%rsp, \%rbp} changes
the base pointer to the top of the stack. The instruction \key{subq
  \$16, \%rsp} moves the stack pointer down to make enough room for
storing variables.  This program just needs one variable ($8$ bytes)
but because the frame size is required to be a multiple of 16 bytes,
it rounds to 16 bytes.

The next four instructions carry out the work of computing
$\BINOP{+}{52}{\UNIOP{-}{10} }$. The first instruction \key{movq \$10,
  -8(\%rbp)} stores $10$ in variable $1$. The instruction \key{negq
  -8(\%rbp)} changes variable $1$ to $-10$. The \key{movq \$52, \%rax}
places $52$ in the register \key{rax} and \key{addq -8(\%rbp), \%rax}
adds the contents of variable $1$ to \key{rax}, at which point
\key{rax} contains $42$.

The last three instructions are the typical \emph{conclusion} of a
procedure.  The \key{addq \$16, \%rsp} instruction moves the stack
pointer back to point at the old base pointer. The amount added here
needs to match the amount that was subtracted in the prelude of the
procedure.  Then \key{popq \%rbp} returns the old base pointer to
\key{rbp} and adds $8$ to the stack pointer.  The \key{retq}
instruction jumps back to the procedure that called this one and
subtracts 8 from the stack pointer.

The compiler will need a convenient representation for manipulating
x86 programs, so we define an abstract syntax for x86 in
Figure~\ref{fig:x86-ast-a}. The \itm{info} field of the \key{program}
AST node is for storing auxilliary information that needs to be
communicated from one pass to the next. The function \key{print-x86}
provided in the supplemental code converts an x86 abstract syntax tree
into the text representation for x86 (Figure~\ref{fig:x86-a}).

\begin{figure}[tbp]
\fbox{
\begin{minipage}{0.96\textwidth}
\[
\begin{array}{lcl}
\Arg &::=&  \INT{\Int} \mid \REG{\itm{register}}
    \mid \STACKLOC{\Int} \\ 
\Instr &::=& (\key{add} \; \Arg\; \Arg) \mid 
      (\key{sub} \; \Arg\; \Arg) \mid 
      (\key{imul} \; \Arg\;\Arg) \mid 
      (\key{neg} \; \Arg) \mid \\
  && (\key{mov} \; \Arg\; \Arg) \mid 
      (\key{call} \; \mathit{label}) \mid
      (\key{push}\;\Arg) \mid (\key{pop}\;\Arg) \mid (\key{ret}) \\
\Prog &::= & (\key{program} \;\itm{info} \; \Instr^{+})
\end{array}
\]
\end{minipage}
}
\caption{Abstract syntax for x86-64 assembly.}
\label{fig:x86-ast-a}
\end{figure}

\section{From $S_0$ to x86-64 through $C_0$}
\label{sec:plan-s0-x86}

To compile one language to another it helps to focus on the
differences between the two languages. It is these differences that
the compiler will need to bridge. What are the differences between
$S_0$ and x86-64 assembly? Here we list some of the most important the
differences.

\begin{enumerate}
\item x86-64 arithmetic instructions typically take two arguments and
  update the second argument in place. In contrast, $S_0$ arithmetic
  operations only read their arguments and produce a new value.

\item An argument to an $S_0$ operator can be any expression, whereas
  x86-64 instructions restrict their arguments to integers, registers,
  and memory locations.

\item An $S_0$ program can have any number of variables whereas x86-64
  has only 16 registers.

\item Variables in $S_0$ can overshadow other variables with the same
  name. The registers and memory locations of x86-64 all have unique
  names.
\end{enumerate}

We ease the challenge of compiling from $S_0$ to x86 by breaking down
the problem into several steps, dealing with the above differences one
at a time. The main question then becomes: in what order to we tackle
these differences? This is often one of the most challenging questions
that a compiler writer must answer because some orderings may be much
more difficult to implement than others. It is difficult to know ahead
of time which orders will be better so often some trial-and-error is
involved. However, we can try to plan ahead and choose the orderings
based on what we find out.

For example, to handle difference \#2 (nested expressions), we shall
introduce new variables and pull apart the nested expressions into a
sequence of assignment statements.  To deal with difference \#3 we
will be replacing variables with registers and/or stack
locations. Thus, it makes sense to deal with \#2 before \#3 so that
\#3 can replace both the original variables and the new ones. Next,
consider where \#1 should fit in. Because it has to do with the format
of x86 instructions, it makes more sense after we have flattened the
nested expressions (\#2). Finally, when should we deal with \#4
(variable overshadowing)?  We shall be solving this problem by
renaming variables to make sure they have unique names. Recall that
our plan for \#2 involves moving nested expressions, which could be
problematic if it changes the shadowing of variables. However, if we
deal with \#4 first, then it will not be an issue.  Thus, we arrive at
the following ordering.
\[
\xymatrix{
4 \ar[r] & 2 \ar[r] & 1 \ar[r] & 3
}
\]

We further simplify the translation from $S_0$ to x86 by identifying
an intermediate language named $C_0$, roughly half-way between $S_0$
and x86, to provide a rest stop along the way. The name $C_0$ comes
from this language being vaguely similar to the $C$ language. The
differences \#4 and \#1, regarding variables and nested expressions,
are handled by the passes \textsf{uniquify} and \textsf{flatten} that
bring us to $C_0$.
\[\large
\xymatrix@=50pt{
  S_0 \ar@/^/[r]^-{\textsf{uniquify}} & 
  S_0 \ar@/^/[r]^-{\textsf{flatten}} &
  C_0 
}
\]

The syntax for $C_0$ is defined in Figure~\ref{fig:c0-syntax}.  The
$C_0$ language supports the same operators as $S_0$ but the arguments
of operators are now restricted to just variables and integers. The
\key{let} construct of $S_0$ is replaced by an assignment statement
and there is a \key{return} construct to specify the return value of
the program. A program consists of a sequence of statements that
include at least one \key{return} statement.

\begin{figure}[htbp]
\[
\begin{array}{lcl}
\Arg &::=& \Int \mid \Var \\
\Exp &::=& \Arg \mid (\Op \; \Arg^{*})\\
\Stmt &::=& \ASSIGN{\Var}{\Exp} \mid \RETURN{\Arg} \\
\Prog & ::= & (\key{program}\;\itm{info}\;\Stmt^{+})
\end{array}
\]
\caption{The $C_0$ intermediate language.}
\label{fig:c0-syntax}
\end{figure}


To get from $C_0$ to x86-64 assembly requires three more steps, which
we discuss below.
\[\large
\xymatrix@=50pt{
  C_0 \ar@/^/[r]^-{\textsf{select\_instr.}}
  & \text{x86}^{*} \ar@/^/[r]^-{\textsf{assign\_homes}} 
  & \text{x86}^{*} \ar@/^/[r]^-{\textsf{patch\_instr.}}
  & \text{x86}
}
\]
We handle difference \#1, concerning the format of arithmetic
instructions, in the \textsf{select\_instructions} pass.  The result
of this pass produces programs consisting of x86-64 instructions that
use variables.
%
As there are only 16 registers, we cannot always map variables to
registers (difference \#3). Fortunately, the stack can grow quite, so
we can map variables to locations on the stack. This is handled in the
\textsf{assign\_homes} pass. The topic of
Chapter~\ref{ch:register-allocation} is implementing a smarter
approach in which we make a best-effort to map variables to registers,
resorting to the stack only when necessary.

The final pass in our journey to x86 handles an indiosycracy of x86
assembly. Many x86 instructions have two arguments but only one of the
arguments may be a memory reference. Because we are mapping variables
to stack locations, many of our generated instructions will violate
this restriction. The purpose of the \textsf{patch\_instructions} pass
is to fix this problem by replacing every bad instruction with a short
sequence of instructions that use the \key{rax} register.

\section{Uniquify Variables}

The purpose of this pass is to make sure that each \key{let} uses a
unique variable name. For example, the \textsf{uniquify} pass could
translate
\[
\LET{x}{32}{ \BINOP{+}{ \LET{x}{10}{x} }{ x } }
\]
to
\[
\LET{x.1}{32}{ \BINOP{+}{ \LET{x.2}{10}{x.2} }{ x.1 } }
\]

We recommend implementing \textsf{uniquify} as a recursive function
that mostly just copies the input program. However, when encountering
a \key{let}, it should generate a unique name for the variable (the
Racket function \key{gensym} is handy for this) and associate the old
name with the new unique name in an association list. The
\textsf{uniquify} function will need to access this association list
when it gets to a variable reference, so we add another paramter to
\textsf{uniquify} for the association list.

\section{Flatten Expressions}

The purpose of the \textsf{flatten} pass is to get rid of nested
expressions, such as the $\UNIOP{-}{10}$ in the following program,
without changing the behavior of the program.
\[
\BINOP{+}{52}{ \UNIOP{-}{10} }
\]
This can be accomplished by introducing a new variable, assigning the
nested expression to the new variable, and then using the new variable
in place of the nested expressions. For example, the above program is
translated to the following one.
\[
\begin{array}{l}
\ASSIGN{ \itm{x} }{ \UNIOP{-}{10} } \\
\RETURN{ \BINOP{+}{52}{ \itm{x} } }
\end{array}
\]

We recommend implementing \textsf{flatten} as a recursive function
that returns two things, 1) the newly flattened expression, and 2) a
list of assignment statements, one for each of the new variables
introduced while flattening the expression.

Take special care for programs such as the following that initialize
variables with integers or other variables.
\[
\LET{a}{42}{ \LET{b}{a}{ b }}
\]
This program should be translated to 
\[
\ASSIGN{a}{42} \;
\ASSIGN{b}{a} \;
\RETURN{b}
\]
and not the following, which could result from a naive implementation
of \textsf{flatten}.
\[
\ASSIGN{x.1}{42}\;
\ASSIGN{a}{x.1}\;
\ASSIGN{x.2}{a}\;
\ASSIGN{b}{x.2}\;
\RETURN{b}
\]

\section{Select Instructions}

In the \textsf{select\_instructions} pass we begin the work of
translating from $C_0$ to x86. The target language of this pass is a
pseudo-x86 language that still uses variables, so we add an AST node
of the form $\VAR{\itm{var}}$.  The \textsf{select\_instructions} pass
deals with the differing format of arithmetic operations. For example,
in $C_0$ an addition operation could take the following form:
\[
\ASSIGN{x}{ \BINOP{+}{10}{32} }
\]
To translate to x86, we need to express this addition using the
\key{add} instruction that does an inplace update. So we first move
$10$ to $x$ then perform the \key{add}.
\[
(\key{mov}\,\INT{10}\, \VAR{x})\; (\key{add} \;\INT{32}\; \VAR{x})
\]

There are some cases that require special care to avoid generating
needlessly complicated code. If one of the arguments is the same as
the left-hand side of the assignment, then there is no need for the
extra move instruction.  For example, the following
\[
\ASSIGN{x}{ \BINOP{+}{10}{x} }
\quad\text{should translate to}\quad
(\key{add} \; \INT{10}\; \VAR{x})
\]

Regarding the \RETURN{e} statement of $C_0$, we recommend treating it
as an assignment to the \key{rax} register and let the procedure
conclusion handle the transfer of control back to the calling
procedure.

\section{Assign Homes}

As discussed in Section~\ref{sec:plan-s0-x86}, the
\textsf{assign\_homes} pass places all of the variables on the stack.
Consider again the example $S_0$ program $\BINOP{+}{52}{ \UNIOP{-}{10} }$,
which after \textsf{select\_instructions} looks like the following.
\[
\begin{array}{l}
(\key{mov}\;\INT{10}\; \VAR{x})\\
(\key{neg}\; \VAR{x})\\
(\key{mov}\; \INT{52}\; \REG{\itm{rax}})\\
(\key{add}\; \VAR{x} \REG{\itm{rax}})
\end{array}
\]
The one and only variable $x$ is assigned to stack location
\key{-8(\%rbp)}, so the \textsf{assign\_homes} pass translates the
above to
\[
\begin{array}{l}
(\key{mov}\;\INT{10}\; \STACKLOC{{-}8})\\
(\key{neg}\; \STACKLOC{{-}8})\\
(\key{mov}\; \INT{52}\; \REG{\itm{rax}})\\
(\key{add}\; \STACKLOC{{-}8}\; \REG{\itm{rax}})
\end{array}
\]

In the process of assigning stack locations to variables, it is
convenient to compute and store the size of the frame which will be
needed later to generate the procedure conclusion.

\section{Patch Instructions}

The purpose of this pass is to make sure that each instruction adheres
to the restrictions regarding which arguments can be memory
references. For most instructions, the rule is that at most one
argument may be a memory reference.

Consider again the following example.
\[
\LET{a}{42}{ \LET{b}{a}{ b }}
\]
After \textsf{assign\_homes} pass, the above has been translated to
\[
\begin{array}{l}
(\key{mov} \;\INT{42}\; \STACKLOC{{-}8})\\
(\key{mov}\;\STACKLOC{{-}8}\; \STACKLOC{{-}16})\\
(\key{mov}\;\STACKLOC{{-}16}\; \REG{\itm{rax}})
\end{array}
\]
The second \key{mov} instruction is problematic because both arguments
are stack locations. We suggest fixing this problem by moving from the
source to \key{rax} and then from \key{rax} to the destination, as
follows.
\[
\begin{array}{l}
(\key{mov} \;\INT{42}\; \STACKLOC{{-}8})\\
(\key{mov}\;\STACKLOC{{-}8}\; \REG{\itm{rax}})\\
(\key{mov}\;\REG{\itm{rax}}\; \STACKLOC{{-}16})\\
(\key{mov}\;\STACKLOC{{-}16}\; \REG{\itm{rax}})
\end{array}
\]

The \key{imul} instruction is a special case because the destination
argument must be a register.

\section{Testing with Interpreters}

The typical way to test a compiler is to run the generated assembly
code on a diverse set of programs and check whether they behave as
expected. However, when a compiler is structured as our is, with many
passes, when there is an error in the generated assembly code it can
be hard to determine which pass contains the source of the error.  A
good way to isolate the error is to not only test the generated
assembly code but to also test the output of every pass. This requires
having interpreters for all the intermediate languages.  Indeed, the
file \key{interp.rkt} in the supplemental code provides interpreters
for all the intermediate languages described in this book, starting
with interpreters for $S_0$, $C_0$, and x86 (in abstract syntax).

The file \key{run-tests.rkt} automates the process of running the
interpreters on the output programs of each pass and checking their
result.

%%%%%%%%%%%%%%%%%%%%%%%%%%%%%%%%%%%%%%%%%%%%%%%%%%%%%%%%%%%%%%%%%%%%%%%%%%%%%%%%
\chapter{Register Allocation}
\label{ch:register-allocation}

In Chapter~\ref{ch:int-exp} we simplified the generation of x86
assembly by placing all variables on the stack. We can improve the
performance of the generated code considerably if we instead try to
place as many variables as possible into registers.  The CPU can
access a register in a single cycle, whereas accessing the stack can
take from several cycles (to go to cache) to hundreds of cycles (to go
to main memory).  Figure~\ref{fig:reg-eg} shows a program with four
variables that serves as a running example. We show the source program
and also the output of instruction selection. At that point the
program is almost x86 assembly but not quite; it still contains
variables instead of stack locations or registers.

\begin{figure}
\begin{minipage}{0.45\textwidth}
Source program:
\begin{lstlisting}
  (let ([v 1])
  (let ([w 46])
  (let ([x (+ v 7)])
  (let ([y (+ 4 x)])
  (let ([z (+ x w)])
       (- z y))))))
\end{lstlisting}
\end{minipage}
\begin{minipage}{0.45\textwidth}
After instruction selection:
\begin{lstlisting}
  (program (v w x y z)
    (mov (int 1) (var v))
    (mov (int 46) (var w))
    (mov (var v) (var x))
    (add (int 7) (var x))
    (mov (var x) (var y))
    (add (int 4) (var y))
    (mov (var x) (var z))
    (add (var w) (var z))
    (mov (var z) (reg rax))
    (sub (var y) (reg rax)))
\end{lstlisting}
\end{minipage}
\caption{Running example for this chapter.}
\label{fig:reg-eg}
\end{figure}

The goal of register allocation is to fit as many variables into
registers as possible. It is often the case that we have more
variables than registers, so we can't naively map each variable to a
register. Fortunately, it is also common for different variables to be
needed during different periods of time, and in such cases the
variables can be mapped to the same register.  Consider variables $x$
and $y$ in Figure~\ref{fig:reg-eg}.  After the variable $x$ is moved
to $z$ it is no longer needed.  Variable $y$, on the other hand, is
used only after this point, so $x$ and $y$ could share the same
register. The topic of the next section is how we compute where a
variable is needed.


\section{Liveness Analysis}

A variable is \emph{live} if the variable is used at some later point
in the program and there is not an intervening assignment to the
variable.
%
To understand the latter condition, consider the following code
fragment in which there are two writes to $b$. Are $a$ and
$b$ both live at the same time? 
\begin{lstlisting}[numbers=left,numberstyle=\tiny]
(mov (int 5) (var a))    ; @$a \gets 5$@
(mov (int 30) (var b))   ; @$b \gets 30$@
(mov (var a) (var c))    ; @$c \gets x$@
(mov (int 10) (var b))   ; @$b \gets 10$@
(add (var b) (var c))    ; @$c \gets c + b$@
\end{lstlisting}
The answer is no because the value $30$ written to $b$ on line 2 is
never used. The variable $b$ is read on line 5 and there is an
intervening write to $b$ on line 4, so the read on line 5 receives the
value written on line 4, not line 2.

The live variables can be computed by traversing the instruction
sequence back to front (i.e., backwards in execution order).  Let
$I_1,\ldots, I_n$ be the instruction sequence. We write
$L_{\mathsf{after}}(k)$ for the set of live variables after
instruction $I_k$ and $L_{\mathsf{before}}(k)$ for the set of live
variables before instruction $I_k$. The live variables after an
instruction are always the same as the live variables before the next
instruction.
\begin{equation*}
  L_{\mathsf{after}}(k) = L_{\mathsf{before}}(k+1)
\end{equation*}
To start things off, there are no live variables after the last
instruction, so 
\begin{equation*}
  L_{\mathsf{after}}(n) = \emptyset 
\end{equation*}
We then apply the following rule repeatedly, traversing the
instruction sequence back to front.
\begin{equation*}
  L_{\mathtt{before}}(k) = (L_{\mathtt{after}}(k) - W(k)) \cup R(k),
\end{equation*}
where $W(k)$ are the variables written to by instruction $I_k$ and
$R(k)$ are the variables read by instruction $I_k$.
Figure~\ref{fig:live-eg} shows the results of live variables analysis
for the running example. Next to each instruction we write its
$L_{\mathtt{after}}$ set.

\begin{figure}[tbp]
\begin{lstlisting}
  (program (v w x y z)
    (mov (int 1) (var v))      @$\{ v \}$@
    (mov (int 46) (var w))     @$\{ v, w \}$@
    (mov (var v) (var x))      @$\{ w, x \}$@
    (add (int 7) (var x))      @$\{ w, x \}$@
    (mov (var x) (var y))      @$\{ w, x, y\}$@
    (add (int 4) (var y))      @$\{ w, x, y \}$@
    (mov (var x) (var z))      @$\{ w, y, z \}$@
    (add (var w) (var z))      @$\{ y, z \}$@
    (mov (var z) (reg rax))    @$\{ y \}$@
    (sub (var y) (reg rax)))   @$\{\}$@
\end{lstlisting}
\caption{Running example program annotated with live-after sets.}
\label{fig:live-eg}
\end{figure}


\section{Building the Interference Graph}

Based on the liveness analysis, we know the program regions where each
variable is needed.  However, during register allocation, we need to
answer questions of the specific form: are variables $u$ and $v$ ever
live at the same time?  (And therefore cannot be assigned to the same
register.)  To make this question easier to answer, we create an
explicit data structure, an \emph{interference graph}.  An
interference graph is an undirected graph that has an edge between two
variables if they are live at the same time, that is, if they
interfere with each other.

The most obvious way to compute the interference graph is to look at
the set of live variables between each statement in the program, and
add an edge to the graph for every pair of variables in the same set.
This approach is less than ideal for two reasons. First, it can be
rather expensive because it takes $O(n^2)$ time to look at every pair
in a set of $n$ live variables. Second, there is a special case in
which two variables that are live at the same time do not actually
interfere with each other: when they both contain the same value
because we have assigned one to the other.

A better way to compute the edges of the intereference graph is given
by the following rules.

\begin{itemize}
\item If instruction $I_k$ is a move: (\key{mov} $s$\, $d$), then add
  the edge $(d,v)$ for every $v \in L_{\mathsf{after}}(k)$ unless $v =
  d$ or $v = s$.

\item If instruction $I_k$ is not a move but some other arithmetic
  instruction such as (\key{add} $s$\, $d$), then add the edge $(d,v)$
  for every $v \in L_{\mathsf{after}}(k)$ unless $v = d$.
  
\item If instruction $I_k$ is of the form (\key{call}
  $\mathit{label}$), then add an edge $(r,v)$ for every caller-save
  register $r$ and every variable $v \in L_{\mathsf{after}}(k)$.
\end{itemize}

Working from the top to bottom of Figure~\ref{fig:live-eg}, $z$
interferes with $x$, $y$ interferes with $z$, and $w$ interferes with
$y$ and $z$.  The resulting interference graph is shown in
Figure~\ref{fig:interfere}.

\begin{figure}[tbp]
\large
\[
\xymatrix@=40pt{
  v \ar@{-}[r] & w \ar@{-}[r]\ar@{-}[d]\ar@{-}[dr] &  x \ar@{-}[dl]\\
               & y \ar@{-}[r] & z
}
\]
\caption{Interference graph for the running example.}
\label{fig:interfere}
\end{figure}


\section{Graph Coloring via Sudoku}

We now come to the main event, mapping variables to registers (or to
stack locations in the event that we run out of registers).  We need
to make sure not to map two variables to the same register if the two
variables interfere with each other.  In terms of the interference
graph, this means we cannot map adjacent nodes to the same register.
If we think of registers as colors, the register allocation problem
becomes the widely-studied graph coloring
problem~\citep{Balakrishnan:1996ve,Rosen:2002bh}.  

The reader may be more familar with the graph coloring problem then he
or she realizes; the popular game of Sudoku is an instance of the
graph coloring problem. The following describes how to build a graph
out of a Sudoku board.
\begin{itemize}
\item There is one node in the graph for each Sudoku square.
\item There is an edge between two nodes if the corresponding squares
  are in the same row or column, or if the squares are in the same
  $3\times 3$ region.
\item Choose nine colors to correspond to the numbers $1$ to $9$.
\item Based on the initial assignment of numbers to squares in the
  Sudoku board, assign the corresponding colors to the corresponding
  nodes in the graph.
\end{itemize}
If you can color the remaining nodes in the graph with the nine
colors, then you've also solved the corresponding game of Sudoku.

Given that Sudoku is graph coloring, one can use Sudoku strategies to
come up with an algorithm for allocating registers. For example, one
of the basic techniques for Sudoku is Pencil Marks. The idea is that
you use a process of elimination to determine what numbers still make
sense for a square, and write down those numbers in the square
(writing very small). At first, each number might be a
possibility, but as the board fills up, more and more of the
possibilities are crossed off (or erased). For example, if the number
$1$ is assigned to a square, then by process of elimination, you can
cross off the $1$ pencil mark from all the squares in the same row,
column, and region. Many Sudoku computer games provide automatic
support for Pencil Marks. This heuristic also reduces the degree of
branching in the search tree.

The Pencil Marks technique corresponds to the notion of color
\emph{saturation} due to \cite{Brelaz:1979eu}.  The
saturation of a node, in Sudoku terms, is the number of possibilities
that have been crossed off using the process of elimination mentioned
above. In graph terminology, we have the following definition:
\begin{equation*}
  \mathrm{saturation}(u) = |\{ c \;|\; \exists v. v \in \mathrm{Adj}(u) 
     \text{ and } \mathrm{color}(v) = c \}|
\end{equation*}
where $\mathrm{Adj}(u)$ is the set of nodes adjacent to $u$ and
the notation $|S|$ stands for the size of the set $S$.

Using the Pencil Marks technique leads to a simple strategy for
filling in numbers: if there is a square with only one possible number
left, then write down that number! But what if there are no squares
with only one possibility left? One brute-force approach is to just
make a guess. If that guess ultimately leads to a solution, great.  If
not, backtrack to the guess and make a different guess.  Of course,
this is horribly time consuming. One standard way to reduce the amount
of backtracking is to use the most-constrained-first heuristic. That
is, when making a guess, always choose a square with the fewest
possibilities left (the node with the highest saturation).  The idea
is that choosing highly constrained squares earlier rather than later
is better because later there may not be any possibilities left.

In some sense, register allocation is easier than Sudoku because we
can always cheat and add more numbers by spilling variables to the
stack. Also, we'd like to minimize the time needed to color the graph,
and backtracking is expensive. Thus, it makes sense to keep the
most-constrained-first heuristic but drop the backtracking in favor of
greedy search (guess and just keep going).
Figure~\ref{fig:satur-algo} gives the pseudo-code for this simple
greedy algorithm for register allocation based on saturation and the
most-constrained-first heuristic, which is roughly equivalent to the
DSATUR algorithm of \cite{Brelaz:1979eu} (also known as
saturation degree ordering
(SDO)~\citep{Gebremedhin:1999fk,Omari:2006uq}).  Just as in Sudoku,
the algorithm represents colors with integers, with the first $k$
colors corresponding to the $k$ registers in a given machine and the
rest of the integers corresponding to stack locations.

\begin{figure}[btp]
  \centering
\begin{lstlisting}[basicstyle=\rmfamily,deletekeywords={for,from,with,is,not,in,find},morekeywords={while},columns=fullflexible]
Algorithm: DSATUR
Input: a graph @$G$@
Output: an assignment @$\mathrm{color}[v]$@ for each node @$v \in G$@

@$W \gets \mathit{vertices}(G)$@
while @$W \neq \emptyset$@ do
    pick a node @$u$@ from @$W$@ with the highest saturation,
        breaking ties randomly
    find the lowest color @$c$@ that is not in @$\{ \mathrm{color}[v] \;|\; v \in \mathrm{Adj}(v)\}$@
    @$\mathrm{color}[u] \gets c$@
    @$W \gets W - \{u\}$@
\end{lstlisting}
  \caption{Saturation-based greedy graph coloring algorithm.}
  \label{fig:satur-algo}
\end{figure}


With this algorithm in hand, let us return to the running example and
consider how to color the interference graph in
Figure~\ref{fig:interfere}. Initially, all of the nodes are not yet
colored and they are unsaturated, so we annotate each of them with a
dash for their color and an empty set for the saturation.
\[
\xymatrix{
  v:-,\{\} \ar@{-}[r] & w:-,\{\} \ar@{-}[r]\ar@{-}[d]\ar@{-}[dr] &  x:-,\{\} \ar@{-}[dl]\\
               & y:-,\{\} \ar@{-}[r] & z:-,\{\}
}
\]
We select a maximally saturated node and color it $0$. In this case we
have a 5-way tie, so we arbitrarily pick $y$. The color $0$ is no
longer available for $w$, $x$, and $z$ because they interfere with
$y$.
\[
\xymatrix{
  v:-,\{\} \ar@{-}[r] & w:-,\{0\} \ar@{-}[r]\ar@{-}[d]\ar@{-}[dr] &  x:-,\{0\} \ar@{-}[dl]\\
               & y:0,\{\} \ar@{-}[r] & z:-,\{0\}
}
\]
Now we repeat the process, selecting another maximally saturated node.
This time there is a three-way tie between $w$, $x$, and $z$. We color
$w$ with $1$.
\[
\xymatrix{
  v:-,\{1\} \ar@{-}[r] & w:1,\{0\} \ar@{-}[r]\ar@{-}[d]\ar@{-}[dr] &  x:-,\{0,1\} \ar@{-}[dl]\\
               & y:0,\{1\} \ar@{-}[r] & z:-,\{0,1\}
}
\]
The most saturated nodes are now $x$ and $z$. We color $x$ with the
next avialable color which is $2$.
\[
\xymatrix{
  v:-,\{1\} \ar@{-}[r] & w:1,\{0,2\} \ar@{-}[r]\ar@{-}[d]\ar@{-}[dr] &  x:2,\{0,1\} \ar@{-}[dl]\\
               & y:0,\{1,2\} \ar@{-}[r] & z:-,\{0,1\}
}
\]
We have only two nodes left to color, $v$ and $z$, but $z$ is
more highly saturaded, so we color $z$ with $2$.
\[
\xymatrix{
  v:-,\{1\} \ar@{-}[r] & w:1,\{0,2\} \ar@{-}[r]\ar@{-}[d]\ar@{-}[dr] &  x:2,\{0,1\} \ar@{-}[dl]\\
               & y:0,\{1,2\} \ar@{-}[r] & z:2,\{0,1\}
}
\]
The last iteration of the coloring algorithm assigns color $0$ to $v$.
\[
\xymatrix{
  v:0,\{1\} \ar@{-}[r] & w:1,\{0,2\} \ar@{-}[r]\ar@{-}[d]\ar@{-}[dr] &  x:2,\{0,1\} \ar@{-}[dl]\\
               & y:0,\{1,2\} \ar@{-}[r] & z:2,\{0,1\}
}
\]

With the coloring complete, we can finalize assignment of variables to
registers and stack locations. Recall that if we have $k$ registers,
we map the first $k$ colors to registers and the rest to stack
lcoations. Suppose for the moment that we just have one extra register
to use for register allocation, just \key{rbx}. Then the following is
the mapping of colors to registers and stack allocations.
\[
  \{ 0 \mapsto \key{\%rbx}, \; 1 \mapsto \key{-8(\%rbp)}, \; 2 \mapsto \key{-16(\%rbp)}, \ldots \}
\]
Putting this together with the above coloring of the variables, we
arrive at the following assignment.
\[
  \{ v \mapsto \key{\%rbx}, \;
  w \mapsto \key{-8(\%rbp)},  \;
  x \mapsto \key{-16(\%rbp)}, \;
  y \mapsto \key{\%rbx},  \;
  z\mapsto \key{-16(\%rbp)} \}
\]
Applying this assignment to our running example
(Figure~\ref{fig:reg-eg}) yields the following program.
% why frame size of 32? -JGS
\begin{lstlisting}
(program 32
  (mov (int 1) (reg rbx))
  (mov (int 46) (stack-loc -8))
  (mov (reg rbx) (stack-loc -16))
  (add (int 7) (stack-loc -16))
  (mov (stack-loc 16) (reg rbx))
  (add (int 4) (reg rbx))
  (mov (stack-loc -16) (stack-loc -16))
  (add (stack-loc -8) (stack-loc -16))
  (mov (stack-loc -16) (reg rax))
  (sub (reg rbx) (reg rax)))
\end{lstlisting}
This program is almost an x86 program. The remaining step is to apply
the patch instructions pass. In this example, the trivial move of
\key{-16(\%rbp)} to itself is deleted and the addition of
\key{-8(\%rbp)} to \key{-16(\%rbp)} is fixed by going through
\key{\%rax}. The following shows the portion of the program that
changed.
\begin{lstlisting}
  (add (int 4) (reg rbx))
  (mov (stack-loc -8) (reg rax)
  (add (reg rax) (stack-loc -16))
\end{lstlisting}
An overview of all of the passes involved in register allocation is
shown in Figure~\ref{fig:reg-alloc-passes}.

\begin{figure}[tbp]
\[
\xymatrix{
  C_0 \ar@/^/[r]^-{\textsf{select\_instr.}}
    & \text{x86}^{*} \ar[d]^-{\textsf{uncover\_live}} \\
    & \text{x86}^{*} \ar[d]^-{\textsf{build\_interference}} \\
    & \text{x86}^{*} \ar[d]_-{\textsf{allocate\_register}} \\
    & \text{x86}^{*} \ar@/^/[r]^-{\textsf{patch\_instr.}} 
    & \text{x86} 
}
\]
\caption{Diagram of the passes for register allocation.}
\label{fig:reg-alloc-passes}
\end{figure}


%%%%%%%%%%%%%%%%%%%%%%%%%%%%%%%%%%%%%%%%%%%%%%%%%%%%%%%%%%%%%%%%%%%%%%%%%%%%%%%%
\chapter{Booleans, Type Checking, and Control Flow}
\label{ch:bool-types}

\section{The $S_1$ Language}

\begin{figure}[htbp]
\centering
\fbox{
\begin{minipage}{0.85\textwidth}
\[
\begin{array}{lcl}
  \Op  &::=& \ldots \mid \key{and} \mid \key{or} \mid \key{not} \mid \key{eq?} \\
  \Exp &::=& \ldots \mid \key{\#t} \mid \key{\#f} \mid
      \IF{\Exp}{\Exp}{\Exp}
\end{array}
\]
\end{minipage}
}
\caption{The $S_1$ language, an extension of $S_0$
  (Figure~\ref{fig:s0-syntax}).}
\label{fig:s1-syntax}
\end{figure}

\section{Type Checking $S_1$ Programs}

% T ::= Integer | Boolean

It is common practice to specify a type system by writing rules for
each kind of AST node. For example, the rule for \key{if} is:
\begin{quote}
  For any expressions $e_1, e_2, e_3$ and any type $T$, if $e_1$ has
  type \key{bool}, $e_2$ has type $T$, and $e_3$ has type $T$, then
  $\IF{e_1}{e_2}{e_3}$ has type $T$.
\end{quote}
It is also common practice to write rules using a horizontal line,
with the conditions written above the line and the conclusion written
below the line.
\begin{equation*}
  \inference{e_1 \text{ has type } \key{bool} & 
        e_2 \text{ has type } T & e_3 \text{ has type } T}
  {\IF{e_1}{e_2}{e_3} \text{ has type } T}
\end{equation*}
Because the phrase ``has type'' is repeated so often in these type
checking rules, it is abbreviated to just a colon. So the above rule
is abbreviated to the following.
\begin{equation*}
  \inference{e_1 : \key{bool} & e_2 : T & e_3 : T}
            {\IF{e_1}{e_2}{e_3} : T}
\end{equation*}

The $\LET{x}{e_1}{e_2}$ construct poses an interesting challenge. The
variable $x$ is assigned the value of $e_1$ and then $x$ can be used
inside $e_2$. When we get to an occurrence of $x$ inside $e_2$, how do
we know what type the variable should be?  The answer is that we need
a way to map from variable names to types.  Such a mapping is called a
\emph{type environment} (aka. \emph{symbol table}). The capital Greek
letter gamma, written $\Gamma$, is used for referring to type
environments environments. The notation $\Gamma, x : T$ stands for
making a copy of the environment $\Gamma$ and then associating $T$
with the variable $x$ in the new environment.  We write $\Gamma(x)$ to
lookup the associated type for $x$.  The type checking rules for
\key{let} and variables are as follows.
\begin{equation*}
  \inference{e_1 : T_1 \text{ in } \Gamma &
             e_2 : T_2 \text{ in } \Gamma,x:T_1}
            {\LET{x}{e_1}{e_2} : T_2 \text{ in } \Gamma} 
  \qquad
  \inference{\Gamma(x) = T}
            {x : T \text{ in } \Gamma}
\end{equation*}
Type checking has roots in logic, and logicians have a tradition of
writing the environment on the left-hand side and separating it from
the expression with a turn-stile ($\vdash$).  The turn-stile does not
have any intrinsic meaning per se.  It is punctuation that separates
the environment $\Gamma$ from the expression $e$.  So the above typing
rules are written as follows.
\begin{equation*}
  \inference{\Gamma \vdash e_1 : T_1 &
             \Gamma,x:T_1 \vdash e_2 : T_2}
            {\Gamma \vdash \LET{x}{e_1}{e_2} : T_2} 
   \qquad
  \inference{\Gamma(x) = T}
            {\Gamma \vdash x : T}
\end{equation*}
Overall, the statement $\Gamma \vdash e : T$ is an example of what is
called a \emph{judgment}.  In particular, this judgment says, ``In
environment $\Gamma$, expression $e$ has type $T$.''
Figure~\ref{fig:S1-type-system} shows the type checking rules for
$S_1$.

\begin{figure}
\begin{gather*}
  \inference{\Gamma(x) = T}
            {\Gamma \vdash x : T}
   \qquad
  \inference{\Gamma \vdash e_1 : T_1 &
             \Gamma,x:T_1 \vdash e_2 : T_2}
            {\Gamma \vdash \LET{x}{e_1}{e_2} : T_2} 
  \\[2ex]
  \inference{}{\Gamma \vdash n : \key{Integer}}
  \quad
  \inference{\Gamma \vdash e_i : T_i \ ^{\forall i \in 1\ldots n} & \Delta(\Op,T_1,\ldots,T_n) = T}
            {\Gamma \vdash (\Op \; e_1 \ldots e_n) : T}
  \\[2ex]
  \inference{}{\Gamma \vdash \key{\#t} : \key{Boolean}}
  \quad
  \inference{}{\Gamma \vdash \key{\#f} : \key{Boolean}}
  \quad
  \inference{\Gamma \vdash e_1 : \key{bool} \\
             \Gamma \vdash e_2 : T &
             \Gamma \vdash e_3 : T}
  {\Gamma \vdash \IF{e_1}{e_2}{e_3} : T}
\end{gather*}
\caption{Type System for $S_1$.}
\label{fig:S1-type-system}
\end{figure}


\begin{figure}

\begin{align*}
\Delta(\key{+},\key{Integer},\key{Integer}) &= \key{Integer} \\
\Delta(\key{-},\key{Integer},\key{Integer}) &= \key{Integer} \\
\Delta(\key{-},\key{Integer}) &= \key{Integer} \\
\Delta(\key{*},\key{Integer},\key{Integer}) &= \key{Integer} \\
\Delta(\key{read}) &= \key{Integer} \\
\Delta(\key{and},\key{Boolean},\key{Boolean}) &= \key{Boolean} \\
\Delta(\key{or},\key{Boolean},\key{Boolean}) &= \key{Boolean} \\
\Delta(\key{not},\key{Boolean}) &= \key{Boolean} \\
\Delta(\key{eq?},\key{Integer},\key{Integer}) &= \key{Boolean} \\
\Delta(\key{eq?},\key{Boolean},\key{Boolean}) &= \key{Boolean} 
\end{align*}

\caption{Types for the primitives operators.}
\end{figure}


\section{The $C_1$ Language}

\begin{figure}[htbp]
\[
\begin{array}{lcl}
\Arg &::=& \ldots \mid \key{\#t} \mid \key{\#f} \\
\Stmt &::=& \ldots \mid \IF{\Exp}{\Stmt^{*}}{\Stmt^{*}} 
\end{array}
\]
\caption{The $C_1$ intermediate language, an extension of $C_0$
  (Figure~\ref{fig:c0-syntax}).}
\label{fig:c1-syntax}
\end{figure}

\section{Flatten Expressions}

\section{Select Instructions}

\section{Register Allocation}

\section{Patch Instructions}


%%%%%%%%%%%%%%%%%%%%%%%%%%%%%%%%%%%%%%%%%%%%%%%%%%%%%%%%%%%%%%%%%%%%%%%%%%%%%%%%
\chapter{Tuples and Heap Allocation}
\label{ch:tuples}

%%%%%%%%%%%%%%%%%%%%%%%%%%%%%%%%%%%%%%%%%%%%%%%%%%%%%%%%%%%%%%%%%%%%%%%%%%%%%%%%
\chapter{Functions}
\label{ch:functions}


%%%%%%%%%%%%%%%%%%%%%%%%%%%%%%%%%%%%%%%%%%%%%%%%%%%%%%%%%%%%%%%%%%%%%%%%%%%%%%%%
\chapter{Lexically Scoped Functions}
\label{ch:lambdas}


%%%%%%%%%%%%%%%%%%%%%%%%%%%%%%%%%%%%%%%%%%%%%%%%%%%%%%%%%%%%%%%%%%%%%%%%%%%%%%%%
\chapter{Mutable Data}
\label{ch:mutable-data}

%%%%%%%%%%%%%%%%%%%%%%%%%%%%%%%%%%%%%%%%%%%%%%%%%%%%%%%%%%%%%%%%%%%%%%%%%%%%%%%%
\chapter{The Dynamic Type}
\label{ch:type-dynamic}


%%%%%%%%%%%%%%%%%%%%%%%%%%%%%%%%%%%%%%%%%%%%%%%%%%%%%%%%%%%%%%%%%%%%%%%%%%%%%%%%
\chapter{Parametric Polymorphism}
\label{ch:parametric-polymorphism}

%%%%%%%%%%%%%%%%%%%%%%%%%%%%%%%%%%%%%%%%%%%%%%%%%%%%%%%%%%%%%%%%%%%%%%%%%%%%%%%%
\chapter{High-level Optimization}
\label{ch:high-level-optimization}


\bibliographystyle{plainnat}
\bibliography{all}

\end{document}

%%  LocalWords:  Dybvig Waddell Abdulaziz Ghuloum Dipanwita
%%  LocalWords:  Sarkar lcl Matz aa representable
